\chapter{Conclusion}

In this work, we presented ALFRED as a general purpose robotics middleware, that can be used to control a robotic arm to turn it into an assistant in many domains such as the industry, research, and personal life. We explained the workings of the middleware, using Docker and Docker Compose to create a separation of permissions and responsibilities in the Kernel and the Userspace, centered around a fast and robust message bus enabling communication between components. We presented interactions that were designed with ALFRED, and how they make us think differently about robotic arms by turning them into smart assistants or an extension of ourselves. We demonstrated applications that were created for ALFRED and then deployed in the system, thus augmenting its capabilities.

The next step for ALFRED is to refine it into something that can compete with existing middlewares in the robotics space, to make ALFRED into something that truly and completely answers the needs of people, industries and researchers.

One path towards this goal is to expose it to the public for user testing. We plan to see how users develop interactions and applications, and add more drivers and Kernel capabilities according to their needs.

Another path is to add environment analysis. While applications can use the robotic arm's API to detect collisions and prevent unwanted movements, we want this to be a core feature of the system by making the robotic arm able to see its environment and react to it as part of the Kernel. It will make the platform safer and less complicated to develop on.

ALFRED gives the tools necessary to create environments where robotic arms live alongside humans not as replacements, but as partners that increase productivity and quality of life.
