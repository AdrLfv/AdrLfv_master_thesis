\chapter{Conclusion}
 
% TODO Reprendre la Neuroplasticity, performance psychology and learning

This thesis contributes to developing and studying new interfaces that promote learning and practice. It studies how to transform a device through various technologies to make learning a complex subject more attractive, mainly using gamification. We first explored the improvement of a tactile interface through the interactive musical score project. The use of conductive ink has made it possible to differentiate the project from traditional smartphone applications by introducing children to the basics of music theory with innovative technology. This score takes advantage of its digital interactivity, flexibility, and sound generation to position itself as an object that is at once attractive, educational, and intuitive. Future work on the project will need to add buttons to the interface, activating certain features such as changing keys or playing associated music. The electronics should also be redone and reduced to improve the embedding of the device. 

The American Sign Language (ASL) Learning Game Project is an AR application that uses the GOSAI framework's modularity on the DVIC augmented mirror to enable fun and practical learning. The application uses different intrinsic functionalities of the OS, such as augmented reality, a robust backend, a permissive frontend, and its suitability for supporting computer vision projects. Exploiting this capability made implementing a custom ASL recognition model possible, which produces an attractive tutorial/correction application when combined with 3D ASL animations. Future work should improve the learning function of the application. Its current effectiveness related to user information retention needs to be improved. Future fixes and lessons features are relevant to the project. 

Similarly, the singing and theremin practice projects aim to correct a user during musical practice on the augmented mirror. They allow a user's performance correction through a frequency estimation module and a GOSAI pose estimation module. The application adds a dimension of improvement in precision, visualization, and attractiveness of musical performance. However, future work concerns the correction of a singer's posture using the mirror's reflection and calibrating the graphical interface with the assistance of a theremin player. This thesis contributed to studying different ways of stimulating interest and investment in learning and practice. The projects studied here make complex fields more attractive while maintaining teaching effectiveness using innovative technology.

\subsection*{Acknowledgments}

I want to thank my supervisor, Dr. Clément DUHART, for guiding me throughout my projects and, more broadly, introducing me to the world of research and innovation by integrating me into the DVIC. The work environment of this space, combining self-improvement, knowledge sharing, and camaraderie, allowed me to flourish during these two wonderful years.

I thank Dr. Marc TEYSSIER for his valuable advice, particularly in electronics, and his feedback on my various projects. I also thank Dr. Xiao XIAO for her advice on writing this thesis, his supervision on my musical projects, and for sharing her passion for music.

I am grateful to Dr. Yliess HATI for introducing me to and sharing his passion for machine learning and computer graphics, Dr. Grégor JOUET for his valuable courses and software advice, and Ph.D. student Madalina NICOLAE for her friendship over these two years.

I thank Maxime BROUSSART for his shared collaboration, investment, and contributions to GOSAI. Our continuous sharing of new ideas for improving our respective platforms and rivalry have been a real source of inspiration and surpassing myself in my projects.

Finally, I can never thank doctoral student Thomas JULDO enough for his investment in my projects, for all his knowledge sharing, which allowed me to achieve what I do today, for sharing my passions in AI and Computer Graphics, for his mentorship, patience, and friendship during these two years. 