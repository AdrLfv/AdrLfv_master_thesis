\chapter{Introduction}

\section{Context}

The learning-performance distinction is a concept in behaviorism that stresses the difference between the learning of a behavior and actual performance of the behavior. Learning is a change in the ability and potential to do when the performance consists in an execution of the learned behavior. \cite{kantak2012learning}

This distinction is especially important in subjects that involve performing physical movements attached to something else – idea, image, sound. For example, in music or language learning.

The relative persistence of learning is sometime referred to as an enhanced capacity for motor skill performance. \cite{kantak2012learning}

These subjects are traditionally difficult to learn because integration with performance takes a lot of time. Acquiring expertise in the practice of a discipline can be laborious if it is done through repetition. People get bored.
The ability to link an idea, a wished sign, sound, mental image with a movement, or a "physical" sound can be perfected with training. This is called performance psychology. It is the scientific field describing the human ability to translate mental concepts into physical or musical practice.

\section{Field of research}

Domain of gamification looks at how to make people less bored when learning subjects, by introducing playful mechanisms \cite[]{saleem2022gamification}.
In recent years gamification has revolutionized the field of professional training. It has accompanied the digital transformation of training and modernized existing modules. This principle of gamification has brought real advantages to learning mechanisms by combining pleasure and skill acquisition.

The gamification of training corresponds to a set of playful mechanisms designed to "gamify" learning content to personalize the relationship with training. This technique improves learner engagement by arousing their interest. Being immersed in an enjoyable educational experience enrich the memorization process thanks to the emotional triggering provided by the game. The playfulness of training will encourage positive emotions that leads to the improvement of learning. The resulting effect is the engagement and motivation improvement.

Another way to improve engagement is through activating different senses. In HCI field, Augmented Reality and Tangible Interfaces make digital information more immersive by projecting it into the real world, engaging sight, sound, and kinesthesia \cite{seichter2007augmented}.

The AR experience is thriving as a significant trend. Around 2.4 billion people use Augmented Reality on their mobile worldwide in 2023. AR can augment computer-generated graphics into the real environment on screen. The gamification of AR and the education system can make students' attitudes more positive. It makes learning interesting, fun, and effortless and improves collaboration and capabilities.

These paradigms make the link between abstract information and the body more legible.
HCI optimizes the symbiosis between user and technology (Human-Computer Confluence), or how the elements of the human ecosystem cooperate with each other to optimize their interaction with humans. Communication Technology (ICT) can be based on radically new forms of sensing, perception, interaction, and understanding \cite{ferscha2007human}. The particularity of digital learning environments lies in the fact that they can accommodate diverse users’ needs \cite{stephanidis2019seven}.

This thesis takes inspiration from work in gamification, performance psychology, and AR/Tangibles and first provides overview of concepts that this document draws from.

\section{Approach}

We have thus carried out a number of projects within the framework of the study of a user's ability to learn and perform with the help of an augmented or tangible interface.

These projects aim at exploiting certain human ways of interacting to promote the use of different intelligences, and thus the retention of information and the ability to practice efficiently. 

\section{Contributions}

The areas studied are: learning the basics of music theory using an interactive electronic score, practicing music through augmented reality learning applications, and learning and practicing sign language through a video game and an AR training application.