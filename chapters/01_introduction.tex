\chapter{Introduction}

\section{Context}

The learning-performance distinction is a concept in behaviorism that stresses the difference between the learning of a behavior and actual performance of the behavior. Learning is a change in the ability and potential to do when the performance consists of an execution of the learned behavior. \cite{kantak2012learning}

This distinction is significant in subjects involving physical movements attached to something else, an idea, image, or sound, for example, in music or language learning.

The relative persistence of learning is sometimes referred to as an enhanced capacity for motor skill performance. \cite{kantak2012learning}

These subjects are traditionally challenging to learn because integration with performance takes much time. Acquiring expertise in the practice of a discipline through repetition can be laborious. People get bored.
With training, the learner can perfect their ability to link an idea, a wished sign, a sound, a mental image with a movement, or a "physical" sound. Performance psychology is the scientific field describing the human ability to translate mental concepts into physical or musical practice.

\section{Field of research}

The domain of gamification looks at how to make people less bored when learning subjects by introducing playful mechanisms \cite[]{saleem2022gamification}.
In recent years gamification has revolutionized the field of professional training. It has accompanied the digital transformation of training and modernized existing modules. This principle of gamification has brought real advantages to learning mechanisms by combining pleasure and skill acquisition.

The gamification of training corresponds to a set of playful mechanisms designed to "gamify" learning content to personalize the relationship with training. This technique improves learner engagement by arousing their interest. Being immersed in an enjoyable educational experience enrich the memorization process thanks to the emotional trigger provided by the game. The playfulness of training will encourage positive emotions that lead to the improvement of learning. The effect is engagement and motivation improvement.

Another way to improve engagement is through activating different senses. In the HCI field, Augmented Reality and Tangible Interfaces make digital information more immersive by projecting it into the real world, engaging sight, sound, and kinesthesia \cite{seichter2007augmented}.

The Augmented Reality (AR) experience is thriving as a significant trend. Around 2.4 billion people use AR on their mobile worldwide in 2023. AR can augment computer-generated graphics into the natural environment on screen. This augmentation can serve gamification, improving the education system efficacity and making students' attitudes more positive. It makes learning interesting, fun, and effortless, improving collaboration and capabilities.

These paradigms make the link between abstract information and the body more legible.
HCI optimizes the symbiosis between user and technology (Human-Computer Confluence) or how the elements of the human ecosystem cooperate to optimize their interaction with humans. Communication Technology (ICT) can be based on radically new forms of sensing, perception, interaction, and understanding \cite{ferscha2007human}. The particularity of digital learning environments lies in the fact that they can accommodate diverse users' needs \cite{stephanidis2019seven}.

According to Howard Gardner, there are 8 forms of intelligence, each of which has certain preferential strengths. These intelligences are: logical-mathematical, verbal-linguistic, musical-rhythmic, bodily-kinesthetic, visual-spatial, interpersonal, intrapersonal and naturalist-ecological. Multiple intelligences are often used to identify the profiles and intelligences of students, to offer them appropriate support. The use of multiple intelligences in learning or practice allows for the stimulation of different areas of the brain and thus promotes and optimizes the retention of conscious and unconscious information in many ways.
%ajouter une illustration

This thesis takes inspiration from work in gamification, performance psychology, and theory of multiple intelligences linked to AR/Tangibles. This document reflects on how to learn and perform by exploiting these different domains.

\section{Approach}

We have thus carried out several projects within the framework of studying a user's ability to learn and perform with the help of an augmented or tangible interface.

These projects aim to exploit specific human ways of interacting to promote using different intelligences, thus the retention of information and the ability to practice efficiently. 

The devices that users interact with use augmented reality, sound, visual feedback, haptics, and enhanced features through electronics or software to help motivate them.

\section{Contributions}

The areas studied are: learning the basics of music theory using an interactive electronic score, practicing music through augmented reality learning applications, and learning and practicing sign language through a video game and an AR training application.