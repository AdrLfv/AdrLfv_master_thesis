\chapter{Introduction}

\section{Context}

The learning-performance distinction is a concept in behaviorism that stresses the difference between the learning of a behavior and actual performance of the behavior. Learning involves a change in one's ability to perform a behavior. \cite{kantak2012learning}

This distinction is particularly important in subjects that involve physical movements, ideas, images, or sounds, such as music or language learning.
The persistence of learning is often referred to as an enhanced capacity for motor skill performance. \cite{kantak2012learning}

These subjects are traditionally challenging to learn due to the integration required between learning and performance. 
With training, the learner can perfect their ability to link an idea, a wished sign, a sound, a mental image with a movement, or a "physical" sound. Performance psychology is the scientific field describing the human ability to translate mental concepts into physical or musical practice.

\section{Field of research}

Gamification is a domain that seeks to make people learning subjects more engaging by incorportating playful mechanisms \cite[]{saleem2022gamification}.
In recent years, gamification has revolutionized the field of professional training, accompanying the digital transformation of training and modernizing existing modules. The gamification of training involves incorporating playful mechanisms into learning content to personalize the relationship between the learner and the training material.

This technique improves learner engagement by arousing their interest and encouraging positive emotions that lead to an improvement in learning. Being immersed in an enjoyable educational experience enrich the memorization process thanks to the emotional trigger provided by the game. The playfulness of training will encourage positive emotions that lead to the improvement of learning. The effect is engagement and motivation improvement.

Another way to improve engagement is through activating different senses. In the HCI field, Augmented Reality and Tangible Interfaces make digital information more immersive by projecting it into the real world, engaging sight, sound, and kinesthesia \cite{seichter2007augmented}.

The Augmented Reality (AR) experience is thriving as a significant trend. Around 2.4 billion people use AR on their mobile worldwide in 2023. AR can augment computer-generated graphics into the natural environment on screen. This augmentation can serve gamification, improving the education system efficacity and making students' attitudes more positive. It makes learning interesting, fun, and effortless, improving collaboration and capabilities.

These paradigms make the link between abstract information and the body more legible.
HCI optimizes the symbiosis between user and technology (Human-Computer Confluence) or how the elements of the human ecosystem cooperate to optimize their interaction with humans. Communication Technology (ICT) can be based on radically new forms of sensing, perception, interaction, and understanding \cite{ferscha2007human}. The particularity of digital learning environments lies in the fact that they can accommodate diverse users' needs \cite{stephanidis2019seven}.

According to Howard Gardner \cite{gardner2004intelligences}, there are 8 forms of intelligence, each of which has certain preferential strengths. These intelligences are: logical-mathematical, verbal-linguistic, musical-rhythmic, bodily-kinesthetic, visual-spatial, interpersonal, intrapersonal and naturalist. Multiple intelligences are often used to identify the profiles and intelligences of students, to offer them appropriate support. The use of multiple intelligences in learning or practice allows for the stimulation of different areas of the brain and thus promotes and optimizes the retention of conscious and unconscious information in many ways.
%ajouter une illustration

\section{Approach}

This thesis draws inspiration from gamification, performance psychology, and the theory of multiple intelligences, all interconnected with augmented reality and tangible interfaces.
The approach involves several projects that study a user's ability to learn and perform using an augmented or tangible interface. 
These projects aim to exploit specific human ways of interacting to promote the use of different intelligences, leading to the retention of information and the ability to practice efficiently. 
The devices used in these projects incorporate augmented reality, sound, visual feedback, haptics, and enhanced features through electronics or software to help motivate users. 

\section{Contributions}

This thesis focuses on several areas of study, including the basics of music theory using an interactive electronic score, practicing music through augmented reality learning applications, and learning and practicing sign language through a video game and an augmented reality training application. 
The contributions of this work include insights into how to effectively learn and perform using augmented and tangible interfaces, promoting engagement and enhancing the retention of information. 
Additionally, this work highlights the importance of multiple intelligences in learning and practice, which can stimulate different areas of the brain and optimize information retention in various ways. 