\chapter{Introduction}

\section{Context}

In the field of educational technology and human-computer interaction, the distinction between learning and performance forms a critical foundation for understanding how interfaces can enhance both aspects.  

Learning, in essence, is the process of automating activities through repeated exposure and practice. It involves acquiring the knowledge and skills necessary to perform a task efficiently. Acquiring proficiency in a novel discipline may pose challenges attributable to the requisite integration between the processes of learning and performance. 

This goes beyond the mere execution of actions. It encompasses the internalization of information, allowing individuals to reproduce movements or recall information without external prompts.  

With training, the learner can perfect his ability to link an idea, a wished sign, a sound, a mental image with a movement, or a "physical" sound. 

This ability to translate mental concepts into concrete practice is called performance psychology. 

As an illustration, it transcends the ability to simply make a movement or read a sentence, delving into the domain of retaining and reproducing these actions independently \cite{kantak2012learning}. 


\section{Approach} 

This thesis studies the influence of gamified interfaces on user performance and learning. The areas concerned include music theory, singing, theremin, and sign-language. It presents several projects that aim to exploit specific human ways of interacting to promote the use of different intelligences, leading to the retention of information and the ability to practice efficiently.  

The devices used in these projects incorporate tangible interfaces, augmented reality, sound, visual feedback, haptics, and enhanced features through electronics or software to help motivate users. 

\section{Field of research}

This thesis draws inspiration from gamification, performance psychology, and the theory of multiple intelligences, all interconnected with augmented reality and tangible interfaces. 

\paragraph[short]{Gamification} 

Gamification is a domain that seeks to make people learning subjects more engaging by incorportating playful mechanisms \cite[]{saleem2022gamification}. 

In recent years, gamification has revolutionized the field of professional training, accompanying the digital transformation of training and modernizing existing modules. The gamification of training involves incorporating playful mechanisms into learning content to personalize the relationship between the learner and the training material. 

This technique improves learner engagement by arousing their interest and encouraging positive emotions that lead to an improvement in learning. Being immersed in an enjoyable educational experience enrich the memorization process thanks to the emotional trigger provided by the game. The playfulness of training will encourage positive emotions that lead to the improvement of learning. The effect is engagement and motivation improvement. 


\paragraph[short]{Augmented Reality and Tangible Interfaces} 

Another way to improve engagement is through activating different senses. In the HCI field, Augmented Reality and Tangible Interfaces make digital information more immersive by projecting it into the real world, engaging sight, sound, and kinesthesia \cite{seichter2007augmented}. 
 
The Augmented Reality (AR) experience is thriving as a significant trend. Around 2.4 billion people use AR on their mobile worldwide in 2023. AR can augment computer-generated graphics into the natural environment on screen. This augmentation can serve gamification, improving the education system efficacity and making students' attitudes more positive. It makes learning interesting, fun, and effortless, improving collaboration and capabilities. 
 
These paradigms make the link between abstract information and the body more legible. 

HCI optimizes the symbiosis between user and technology (Human-Computer Confluence) or how the elements of the human ecosystem cooperate to optimize their interaction with humans. Communication Technology (ICT) can be based on radically new forms of sensing, perception, interaction, and understanding \cite{ferscha2007human}. The particularity of digital learning environments lies in the fact that they can accommodate diverse users' needs \cite{stephanidis2019seven}. 

\paragraph[short]{Multiple Intelligences and Multimodality}

The primary purpose of an interface is to allow the link with the user to a piece of equipment or an application. It is built with the aim of best optimizing its primary mission. These technologies can be used for the purpose of assisting learning. They must therefore adapt as best as possible to the specificities of human brain capacities for learning (i.e. stimulate cerebral plasticity, synaptogenesis or even neurogenesis). \cite[]{maes1993learning}

According to Howard Gardner \cite{gardner2004intelligences}, there are 8 forms of intelligence, each of which has certain preferential strengths. These intelligences are: logical-mathematical, verbal-linguistic, musical-rhythmic, bodily-kinesthetic, visual-spatial, interpersonal, intrapersonal and naturalist. Multiple intelligences are often used to identify the profiles and intelligences of students, to offer them appropriate support. The use of multiple intelligences in learning or practice allows for the stimulation of different areas of the brain and thus promotes and optimizes the retention of conscious and unconscious information in many ways. 

\section{Approach}

This thesis draws inspiration from gamification, performance psychology, and the theory of multiple intelligences, all interconnected with augmented reality and tangible interfaces.
The approach involves several projects that study a user's ability to learn and perform using an augmented or tangible interface. 
These projects aim to exploit specific human ways of interacting to promote the use of different intelligences, leading to the retention of information and the ability to practice efficiently. 
The devices used in these projects incorporate augmented reality, sound, visual feedback, haptics, and enhanced features through electronics or software to help motivate users. 

\section{Contributions}

This thesis focuses on several areas of study, including the basics of music theory using an interactive electronic score, practicing music through augmented reality learning applications, and learning and practicing sign language through a video game and an augmented reality training application.  

The contributions of this work include insights into how to effectively learn and perform using augmented and tangible interfaces, promoting engagement, and enhancing the retention of information. The presented interfaces are designed to exploit brain plasticity and improve the learning process by integrating complex cognitive mechanisms. By using tangible devices, these interfaces promote sensorimotor engagement, involving the user's senses and physical movements. Gamification, based on principles of positive reinforcement and reward, activates the brain's reward system, releasing neurotransmitters such as dopamine, which are associated with motivation and learning. Synaptogenesis and long-term synaptic plasticity (LTP) are also stimulated, strengthening neuronal connections relevant for information retention. Furthermore, by creating immersive learning environments, these interfaces exploit spatial memory and memory consolidation, thus promoting more effective knowledge retention \cite{janssen2017gamification}. Our contribution creates enriched learning contexts, exploiting the neurobiological bases of the cognitive process to optimize the assimilation and retention of information. 