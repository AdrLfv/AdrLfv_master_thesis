%=============================================
\chapter{Chapter name}

% INFO =============================================
\textbf{~20 pages}

This chapter will present one of your project made during your Master Thesis.
You can change this structure if you feel it's more adapted. Be sure to discuss with you advisor beforehand.
% INFO =============================================

%=============================================
\section{Introduction}

% INFO =============================================
\textbf{~1 page}
\begin{itemize}
    \item What is the context of your work? Present the field in general
    \item Resume briefly the projecy your contribution
\end{itemize}
% INFO =============================================

%=============================================
\section{Related Work}

% INFO =============================================
\textbf{~4 pages}
\begin{itemize}
    \item How did researchers tackle the challenge at hand in the past
    \item What are the trade-offs of the different alternatives and why did you choose a particular one
    \item How does your contribution relate to prior work
    \item Also includes technical state of the art, i.e. available technologies and technologies used in this work
\end{itemize}
% INFO =============================================

%=============================================
% 5-10 research references articulated in a telling storing demonstrating
\section{Contribution}

% INFO =============================================
\textbf{~13 pages}

Make sure to not only report what you have done (this would be a technical manual or whitepaper) but also explain why you have done it. For instance:

\begin{itemize}
    \item Why you have chosen specific technologies and not others?
    \item Why have you realized the interface in this specific way?
    \item Why did you opt for this way of evaluation?
\end{itemize}

Reporting your prototypes, trials and errors are also valuable.

% Concept
\textbf{~1 pages}
What did you do to push the state of the art?

% Implementation
\textbf{~7 pages}

Go into details, don't lose the bigger picture. Share details of the implementation, fabrications, different states of the prototype. Provide high qualities pictures of your prototype, schemas of the architecture or everything that is relevant for your work.

% Evalutation
\textbf{~2 pages or more}

This section includes a quantitative or qualitative evaluation (user studies or experimentation)
To write this section, draw inspiration from the evaluation section of the scientific papers! Describe the goal, your protocol, and the results with graphs. Comments the results. How different is your work from related work / other similar projects.
% INFO =============================================

\subsection{Concept}
\subsection{Implementation}
\subsection{Evaluation}

%=============================================
\section{Conclusion}

% INFO =============================================
\textbf{~2 pages}
\begin{itemize}
    \item An overview of the project
    \item What did you learn from the project that could benefit others
    \item What can you do that you could not before?
    \item What would be the required steps that had to be removed from the scope?
\end{itemize}
% INFO =============================================

\subsection{Applications and use cases}
\subsection{Limitations}
\subsection{Future works}
